\documentclass[a4paper,10pt]{report}
\usepackage[utf8]{inputenc}
\usepackage{amsmath}
\usepackage{graphicx}
\usepackage{caption}
\usepackage{subcaption}
\usepackage[usenames,dvipsnames,svgnames,table]{xcolor}
\usepackage{framed,color}
\usepackage{listings}

\definecolor{mygreen}{rgb}{0,0.6,0}
\definecolor{mygray}{rgb}{0.5,0.5,0.5}
\definecolor{mymauve}{rgb}{0.58,0,0.82}

\lstset{ %
  backgroundcolor=\color{white},   % choose the background color; you must add \usepackage{color} or \usepackage{xcolor}
  basicstyle=\footnotesize,        % the size of the fonts that are used for the code
  breakatwhitespace=false,         % sets if automatic breaks should only happen at whitespace
  breaklines=true,                 % sets automatic line breaking
  captionpos=b,                    % sets the caption-position to bottom
  commentstyle=\color{mygreen},    % comment style
  deletekeywords={...},            % if you want to delete keywords from the given language
  escapeinside={\%*}{*)},          % if you want to add LaTeX within your code
  extendedchars=true,              % lets you use non-ASCII characters; for 8-bits encodings only, does not work with UTF-8
  frame=single,                    % adds a frame around the code
  keepspaces=true,                 % keeps spaces in text, useful for keeping indentation of code (possibly needs columns=flexible)
  keywordstyle=\color{blue},       % keyword style
  language=Octave,                 % the language of the code
  morekeywords={*,...},            % if you want to add more keywords to the set
  numbers=left,                    % where to put the line-numbers; possible values are (none, left, right)
  numbersep=5pt,                   % how far the line-numbers are from the code
  numberstyle=\tiny\color{mygray}, % the style that is used for the line-numbers
  rulecolor=\color{black},         % if not set, the frame-color may be changed on line-breaks within not-black text (e.g. comments (green here))
  showspaces=false,                % show spaces everywhere adding particular underscores; it overrides 'showstringspaces'
  showstringspaces=false,          % underline spaces within strings only
  showtabs=false,                  % show tabs within strings adding particular underscores
  stepnumber=2,                    % the step between two line-numbers. If it's 1, each line will be numbered
  stringstyle=\color{mymauve},     % string literal style
  tabsize=2,                       % sets default tabsize to 2 spaces
  title=\lstname                   % show the filename of files included with \lstinputlisting; also try caption instead of title
}

% Title Page
\title{Telecom Software lab EEP 773\\Assignment 9\\{\bf \color{blue}Learning Python and git}}
\author{\emph{Submitted by:}\\Nitin K. Lohar\\2012EEZ8074}

\begin{document}
\maketitle
\tableofcontents
%\listoffigures

\chapter{Problem statement}
Write python program to find emotions of users based on the different smileys occurring in the
text and looking up emotion of that smiley from a given dictionary. Also, output percentage of
each emotion occurring in the text.

\chapter{Solution}
We have written python script for the problem. Script written is given in later sections.

\section{Assumptions}
We assume that smiley can occur only after a white space or a full stop. Smiley occurring just after a word (without a
space) is not considered as a smiley. Although, there can be a word just after the smiley. Similarly, we assume that ;);)
represents sarcastic mood. Any other smiley repeating without space or dot is ignored.\\
While deciding mood of a user, we count number of occurrence for each mood and then whichever mood has highest count, we assume 
that mood. If two mood have same max occurrence, we choose randomly.

\section{Logic}
We have created a dictionary first using the file ``mood\_dict.txt''. After this, we read line by line contents of ``contents.txt'' file
and search for regular expression of smileys. Then we add those occurrences to the user database. Finally, we sort the user database according
to the occurrence of mood and write the max occurred mood to the file. Then we write occurrence of each mood in the file in percentage.

\chapter{Results}
The code files and output files are attached. Also, same have been pushed to the github account.

\section{Python code for the task}
\lstinputlisting[language=python]{assgn9.py}

\end{document}          